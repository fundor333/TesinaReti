\documentclass{report}

\usepackage[T1]{fontenc}	% imposta la codifica dei font
\usepackage[utf8]{inputenc}	% lettere accentate da tastiera
\usepackage[italian]{babel}	% per scrivere in italiano
\usepackage[babel]{csquotes}

%*********************************************************************************
% Impostazioni di biblatex
%*********************************************************************************
\usepackage[bibstyle=numeric,citestyle=numeric,backend=bibtex]{biblatex}

\bibliography{bibliografiareti} 


\DeclareBibliographyCategory{libri}
\DeclareBibliographyCategory{web}

\defbibheading{libri}{\subsection*{Manuali e Testi consultati}}
\defbibheading{web}{\subsection*{Siti Web consultati}}


%*********************************************************************************
% Fine impostazioni di biblatex
%*********************************************************************************


\usepackage{graphicx}
\title{\includegraphics[width=120mm]{logo-unive.png} \\ Xml e Ajax}
\author{Matteo Scarpa\\ 845087}
\date{}

\begin{document}

\maketitle
\tableofcontents

\chapter{XML}
\section{Cos'è un file XML}
XML è un \textit{linguaggio di markup} e XML sta per e\textbf{X}tensible \textbf{M}arkup \textbf{L}anguage\footnote{Standard W3C \cite{site:w3cxml}} ovvero un linguaggio che consente di definire gli elementi di un documento attraverso meccanismi sintattici.
È derivato dallo standard SGML  ed è pensato per formattare informazioni ed è nato per uso legato alla rete anche se poi è stato esteso anche a formato per il salvataggio di file.

\subsection{Storia del XML}
XML si prefigge di essere una versione semplificata di \textbf{S}tandard \textbf{G}eneralized \textbf{M}arkup \textbf{L}anguage (conosciuto anche come SGML o ISO 8879) da cui deriva.
Il SGML è un metalinguaggio definito come standard ISO\footnote{Standard ISO \cite{site:isosgml}}  avente lo scopo di definire linguaggi da utilizzare per la stesura di testi destinati ad essere trasmessi ed archiviati con strumenti informatici, ossia per la stesura di documenti in forma leggibile da computer (machine readable form). Da questo linguaggio XML prende il DTD (o Document Type Definition) che è un documento che definisce le componenti ammesse e come devono essere strutturati.

Lo sviluppo dell' XML parte nel 1996 quando si costituì l’XML Working Group all'interno del W3C per poter definire un linguaggio di markup che offrisse maggior libertà nella definizione dei tag rispetto a SGML. Questo viene fatto perchè l'assenza di uno standard abbastanza elastico stava portando i vari sviluppatori di browser e di formati per il web a fare "estensioni" non standard creando caos nella condivisione dei file nel web in quanto solo il loro prodotto poteva visualizzare correttamente i file.
L' XML Working Group ha quindi definito la struttura di base dell' XML e ha portato alla sua prima definizione di uno standard riconosciuto dell XML. 


Questo standar, inizialmente pensato per la trasmissione dei dati attraverso il web, ha avuto risvolti inaspettati. Dopo poco tempo ha infatti permesso lo sviluppo di tutta una serie di formati di file indipendenti dalla trasmissione via web, pensati per il semplice salvataggio.
Per questo motivo spesso il formato XML viene considerato appartenente al livello di presentazione del modello ISO.\footnote{come è scritto anche nel\textcite[p.~42]{tanenbaum:reti} il livello di presentazione come quello di codifica della sintassi dei file per questo motivo viene spesso associato a questo livello}

Tra i vari formati derivati ricordiamo:
\begin{itemize}
\item \textbf{ODF} formato di documenti OpenSource. Si struttura come una cartella compressa contenente file XML per la definizione del contenuto del documento\footnote{Standard ISO \cite{site:isoodf} }
\item \textbf{SVG} formato per le immagini vettoriali. Definisce gli oggetti geometrici attraverso funzioni matematiche. Questo permette di avere immagini "leggibili" da codice e facilemente modificabili con processi automatici, che vanno a leggere il contenuto dell'XML e a modificare i tag di interesse.\footnote{Standard  W3C \cite{site:w3csvg}}
\item \textbf{X3D} formato per gli oggetti tridimensionali e delle scene in cui tali oggetti esistono. Utilizzato da alcuni programmi di grafica 3D per la visualizzazione web \footnote{Standard W3C \cite{site:w3dx3d}}
\end{itemize}
In oltre molti database strutturano i dati in formato XML perchè è più facile elaborarli in tempi costanti e risulta più facilmente adattabile al paradigma del database funzionale.

\subsection{Struttura}
La struttura di un file XML è sempre la stessa:
\begin{itemize}
\item Un \textbf{header}, solitamente sotto la forma di un tag XML,  che indica la versione XML (o derivate) utilizzata per la stesura del documento e la versione della codifica utilizzata nel documento XML
\item Un \textbf{tag di corpo} contenente tutti gli altri tag strutturati ad albero. In oltre, nel tag di corpo c'è la possibilità che sia anche presente la definizione del namespace utilizzato per la definizione dei tag.
\end{itemize}
In più, per essere correttamente visualizzato deve essere \textbf{ben formattato} ovvero
\begin{itemize}
\item Ogni etichetta o tag deve essere compreso tra due parentesi angolate (<>) e deve avere il corrispettivo tag di chiusura. Questo può essere realizzato attraverso una coppia di tag (<\textit{etichetta}></ \textit{etichetta}>) o un tag unico che si chiude in se stesso (<\textit{etichetta}/>)
\item I tag non possono iniziare con numeri o caratteri speciali
\item I tag non possono contenere spazi (solitamente vengono sostituiti con \_ o semplicemente non vengono messi)
\item I commenti devono essere definiti nel tag di commento (<!--~\textit{corpo~del~commento}~-->) e possono essere posizionati in ogni punto del documento
\item I tag devono essere correttamente annidati (vanno chiusi nell'ordine inverso di come sono stati aperti)
\item I tag sono \textit{case sensitive} quindi bisogna fare attenzione quando definisco i tag (esempio: il tag \textit{PianoForte} è diverso dal tag \textit{Pianoforte} e possono essere utilizzati per indicare tag dalla funzione diversa)
\end{itemize}

\section{Grammatica e definizione degli elementi sintattici}
Il formato di file XML per certi punti di vista è molto aperto e permette una grande libertà nei tag pur mantenendo una sintassi stretta e bene definita.
Questo comporta, però, una definizione di sintassi unica per la rappresentazione dei documenti che bisogna concordare tra gli utilizzatori di quel determinato tipo di documento XML. Questo ha portato alla necessità di definire delle \textbf{grammatiche} ovvero dei documenti che definisco quali sono i tag XML da utilizzare e come all'interno del documento che si ha intenzione di utilizzare.


Questa tipologia di documenti viene definita in un file (solitamente a parte rispetto all' XML e facilmente reperibile come può essere con un indirizzo internet) che viene inserito sottoforma di link all'interno dell XML in cui vengono inserite le informazioni. Questo documento solitamente esprime dei vincoli su come debba essere strutturato XML: 
\begin{itemize}
\item Qual'è il tag radice
\item Quali e quanti tag sono sotto a un determinato tag
\item Quali attributi possono avere i vari tag
\item Quali tag non possono mancare 
\end{itemize}

Per questo genere di lavoro sono stati definiti due tipi di file: DTD e il XML Schema 

\subsection{Document Type Definition}
Il \textbf{D}ocument \textbf{T}ype \textbf{D}efinition (o DTD) permette di descrivere i tag html da utilizzare all'interno di un documento e il loro rapporto reciproco. Questo permette un ampio controllo sul contenuto del documento e semplifica il lavoro di elaborazione di entrambi le parti. 

\section{XML Schema}
Successivamente per le definizioni di grammatiche per le applicazioni XML è stato in trodotto il linguaggio XML Schema, pensato apposta per questa funzione. Questo nuovo tipo di documenti permette un maggior controllo nella definizione dei documenti Xml attraverso i namespace ed è l'unico linguaggio di definizione del contenuto per XML approvato dalla W3C (\cite{site:w3cxmlschame} )


\chapter{AJAX}
AJAX è una tecnica di sviluppo software basato sullo scambio dati in backgound tra il client e il server.  Il nome è un acronimo che sta per \textbf{A}synchronous \textbf{J}avaScript \textbf{A}nd \textbf{X}ML spiega anche come è nato.

AJAX nasce come script che permette lo scambio di dati asincrono tra il client e il server, evitando così la richiesta al server da parte del browser e permettendo all'utente di non aggiornare la pagina visualizzata pur avendo una interazione col server. Nonostante il nome suggerisca l'uso di due tecnologie (Javascript e XML per l'appunto) questo non vuole dire che AJAX si limita solo a queste due ma utilizza anche altre tecnologie quali JSON, JScript e VBScript.

Questa tecnica di sviluppo è largamente utilizzata nelle applicazioni web con poca o nessuna richiesta di refresh della pagina ma comunque la necessità di inviare e/o ricevere dati dal server. Spesso infatti si hanno applicazioni che recuperano in tempo reale le informazioni e "aggironano" la pagina con i nuovi dati senza dover riscrivere tutto.

\section{XMLHttp} Per realizzare un applicativo AJAX è necessario utilizzare un API o una libreria come XMLHttp che ti permetta di aggiornare la pagina dinamicamente richiedendo i dati al server sotto forma di file XML.\footnote{Ovviamente dal nome è facile capire che questa API permette di lavorare senza problemi sopratutto se il server risponde con file XML anche se supporta il JSON e i testi semplici} 
La maggior parte dei browser supportano l'API XMLHttp\footnote{vedi \textcites{site:xmlhttpcompatibilita}}

\newpage
\section*{\refname}
\printbibliography[heading=web, nottype=book]
\printbibliography[heading=libri, type=book]

\end{document}