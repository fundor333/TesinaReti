\documentclass{beamer}

\usepackage[T1]{fontenc}	% imposta la codifica dei font
\usepackage[utf8]{inputenc}	% lettere accentate da tastiera
\usepackage[italian]{babel}	% per scrivere in italiano
\usepackage{listings}

\title{Xml e Ajax}
\author{Matteo Scarpa\\ 845087}
\date{}
\institute{Università Ca'Foscari}
\logo{\includegraphics[width=15mm]{logo-unive.png}}

\usetheme{Szeged}
\usecolortheme{beaver}

\begin{document}

\begin{frame}
   \maketitle
\end{frame}

\section{XML}

\subsection{XML}
\begin{frame}
    \frametitle{Cos'è XML}
    \begin{itemize}
        \item <1->e\textbf{X}tensible \textbf{M}arkup \textbf{L}anguage
        \item <2->Linguaggio di markup
        \item <3->È derivato dallo standard SGML (ISO 8879)
        \item <4->Viene usato per salvare o formattare informazioni
        \item <5->Viene affiancato da DTD e XML Schema
    \end{itemize}
\end{frame}

\begin{frame}
    \frametitle{Come è nato l'XML}
    \begin{itemize}
    \item Deriva da SGML, un linguaggio di markup
    \item Inizia ad essere sviluppato nel 1996
    \item Definito come standard nel 1998
    \item Viene creato per definire uno standard
    \item Inizialmente è pensato solo per formattare i dati per la rete
    \end{itemize}

\end{frame}

\begin{frame}
    \frametitle{Struttura del XML}
    \begin{itemize}
        \item Header
        \item Tag principale
        \begin{itemize}
            \item Tag secondari
            \item Tag secondari annidati
        \end{itemize}
    \end{itemize}
\end{frame}

\begin{frame}
    \frametitle{Vincoli sintattici}
    Per poter passare la validazione i tag devono:
    \begin{itemize}
    \item <2->Essere compreso tra due parentesi angolate 
    \item <3->Essere chiuso con \/ o avere il tag di chiusura
    \item <4->Essere correttamente annidati 
    \item <5->Non possono iniziare con numeri 
    \item <6->Non possono iniziare con caratteri speciali
    \item <7->Non possono contenere spazi 
    \end{itemize}
\end{frame}


\begin{frame}[fragile]
    \frametitle{Esempio di XML}
    \begin{verbatim}
<note>
    <to>Mamma</to> 
    <from>Matteo</from> 
    <heading>Cena</heading> 
    <body>Dormo da amici e non ceno a casa. Baci Teo</body> 
</note> 
    \end{verbatim}
\end{frame}

\begin{frame}
    \frametitle{XSLT}
    \begin{itemize}
    \item <1->E\textbf{X}tensible \textbf{S}tylesheet \textbf{L}anguage \textbf{T}ransformations
    \item <2-> Trasforma XML in formato più leggibile
    \item <3-> Più pratico dell' elaborazioni via script
    \item <4-> Output in XHTML e XML elaborato
    \end{itemize}
\end{frame}

\subsection{Grammatiche}
\begin{frame}
    \frametitle{Funzionalità della grammatica}
    Definisce
    \begin{itemize}
        \item<2-> Qual'è il tag radice
        \item<3-> Quali e quanti tag sono sotto a un determinato tag
        \item<4-> Quali attributi possono avere i vari tag
        \item<5-> Quali tag non possono mancare 
        \item<6-> Valida il documento XML
    \end{itemize}
\end{frame}

\begin{frame}
    \frametitle{DTD}
    \begin{itemize}
    \item <1-> \textbf{D}ocument \textbf{T}ype \textbf{D}efinition
    \item <2-> Definisce vincoli sintattici del documento
    \item <3-> Deriva dal SGML
    \item <5-> Non è lo standard per XML
    \end{itemize}
\end{frame}

\begin{frame}
\frametitle{XML Schema}
\begin{itemize}
    \item <1-> Standard ufficiale W3C per definire la sintassi XML
    \item <2-> Utilizza i namespace per definire i tag
    \item <3-> Più efficente del DTD in quanto pensato per XML
\end{itemize}
\end{frame}

\begin{frame}[fragile]
\frametitle{Esempio di XMLS}
\begin{verbatim}
<?xml version="1.0"?>
<xs:schema xmlns:xs="http://www.w3.org/2001/XMLSchema">
<xs:element name="note">
  <xs:complexType>
    <xs:sequence>
      <xs:element name="to" type="xs:string"/>
      <xs:element name="from" type="xs:string"/>
      <xs:element name="heading" type="xs:string"/>
      <xs:element name="body" type="xs:string"/>
    </xs:sequence>
  </xs:complexType>
</xs:element>
</xs:schema>
\end{verbatim}
\end{frame}


\section{AJAX}
\subsection{Ajax}
\begin{frame}
    \frametitle{Cos'è Ajax}
    \begin{itemize}
    \item <1-> \textbf{A}synchronous \textbf{J}avaScript \textbf{A}nd \textbf{X}ML
    \item <2-> Metodo di programmazione 
    \item <3-> Prevede uno script e un server che elabori la richiesta
    \item <4-> Non si usa sempre Javascript e XML
    \end{itemize}
\end{frame}

\begin{frame}
    \frametitle{XMLHttp}
    \begin{itemize}
    \item <1-> API per fare e ricevere richieste HTTP
    \item <2-> Lavora per lo più con il formato XML
    \item <3-> Standard "de facto" per le richieste da script
    \end{itemize}

\end{frame}


\subsection{Vantaggi e Svantaggi}
\begin{frame}
    \frametitle{Vantaggi AJAX}
    \begin{itemize}
    \item<1-> Permette di caricare la struttura della pagina senza i dati    
    \item<2-> Permette di aggiornare i dati in tempo reale
    \item<3-> Permette di riciclare le pagine
    \end{itemize}
\end{frame}

\begin{frame}
    \frametitle{Problemi di AJAX}
    \begin{itemize}
    \item<1-> La cronologia non registra i cambiamenti delle pagine prodotti con AJAX
    \begin{itemize}
        \item <3-> HTML5 supporta l'inserimento di eventi nella cronologia
        \item <3-> Impostabile da script
    \end{itemize}
    \item<2-> Le pagine AJAX necessitano di un linguaggio di scripting per funzionare
    \begin{itemize}
        \item <4-> Usare solo linguaggi ampiamente diffusi
        \item <4-> Posizionare testo alternativo in assenza di script
    \end{itemize}
    \end{itemize}
\end{frame}


\end{document}