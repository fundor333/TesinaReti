\documentclass{beamer}

\usepackage[T1]{fontenc}	% imposta la codifica dei font
\usepackage[utf8]{inputenc}	% lettere accentate da tastiera
\usepackage[italian]{babel}	% per scrivere in italiano
\usepackage{listings}

\title{Xml e Ajax}
\author{Matteo Scarpa\\ 845087}
\date{}
\institute{Università Ca'Foscari}
\logo{\includegraphics[width=15mm]{logo-unive.png}}

\usetheme{Szeged}
\usecolortheme{beaver}

\begin{document}

\begin{frame}
   \maketitle
\end{frame}

\section{XML}

\subsection{Cos'è un file XML}
\begin{frame}
    \frametitle{Cos'è XML}
    \begin{itemize}
        \item Linguaggio di markup
        \item e\textbf{X}tensible \textbf{M}arkup \textbf{L}anguage
        \item È derivato dallo standard SGML (ISO 8879)
        \item Viene usato per salvare o formattare informazioni
        \item Viene affiancato da DTD e XML Schema
    \end{itemize}
\end{frame}

\subsection{Storia del XML}
\begin{frame}
    \begin{itemize}
    \item Deriva da SGML, un linguaggio di markup
    \item Viene creato durante la guerra dei browser
    \item Inizialmente è pensato solo per formattare i dati per l'invio
    \end{itemize}

\end{frame}



\subsection{Esempio di XML}
\begin{frame}[fragile]
    \frametitle{Esempio di XML}
    \begin{verbatim}
<DataModelElement xsi:type="webml:Entity" id="0" >       
    <Attribute id="1" name="OID" type="OID" /> 
    <Attribute id="2" name="Name"/> 
    <Attribute id="3" name="Psw"/> 
    <Attribute id="4" name="Email" contentType=""/> 
</DataModelElement> 
    \end{verbatim}
\end{frame}

\subsection{XSLT}
\subsection{Esempio di output di XSLT}
\subsection{Document Type Definition}
\subsection{XML Schema}
\subsubsection{Esempio di XMLS}

\section{AJAX}
\subsection{Cos'è Ajax}

\begin{frame}
    \frametitle{Cos'è Ajax}
    \begin{itemize}
    \item \textbf{A}synchronous \textbf{J}avaScript \textbf{A}nd \textbf{X}ML
    \end{itemize}

\end{frame}

\subsection{Problemi e vantaggi derivanti da AJAX}
\subsection{XMLHttp} 

\end{document}