\documentclass{beamer}

\usepackage[T1]{fontenc}	% imposta la codifica dei font
\usepackage[utf8]{inputenc}	% lettere accentate da tastiera
\usepackage[italian]{babel}	% per scrivere in italiano
\usepackage{listings}

\title{Xml e Ajax}
\author{Matteo Scarpa\\ 845087}
\date{}
\institute{Università Ca'Foscari}
\logo{\includegraphics[width=15mm]{logo-unive.png}}

\usetheme{Szeged}
\usecolortheme{beaver}

\begin{document}

\begin{frame}
   \maketitle
\end{frame}

\section{XML}

\subsection{Cos'è un file XML}
\begin{frame}
    \frametitle{Cos'è XML}
    \begin{itemize}
        \item <1->e\textbf{X}tensible \textbf{M}arkup \textbf{L}anguage
        \item <2->Linguaggio di markup
        \item <3->È derivato dallo standard SGML (ISO 8879)
        \item <4->Viene usato per salvare o formattare informazioni
        \item <5->Viene affiancato da DTD e XML Schema
    \end{itemize}
\end{frame}

\begin{frame}
    \frametitle{Come è nato l'XML}
    \begin{itemize}
    \item Deriva da SGML, un linguaggio di markup
    \item Inizia ad essere sviluppato nel 1996
    \item Definito come standard nel 1998
    \item Viene creato per definire uno standard
    \item Inizialmente è pensato solo per formattare i dati per la rete
    \end{itemize}

\end{frame}

\subsection{Struttura di un XML}
\begin{frame}
    \frametitle{Struttura del XML}
    \begin{itemize}
        \item Header
        \item Tag principale
        \begin{itemize}
            \item Tag secondari
            \item Tag secondari annidati
        \end{itemize}
    \end{itemize}
\end{frame}

\begin{frame}
    \frametitle{Vincoli sintattici}
    Per poter passare la validazione i tag devono:
    \begin{itemize}
    \item <2->Essere compreso tra due parentesi angolate 
    \item <3->Essere chiuso con \/ o avere il tag di chiusura
    \item <4->Essere correttamente annidati 
    \item <5->Non possono iniziare con numeri 
    \item <6->Non possono iniziare con caratteri speciali
    \item <7->Non possono contenere spazi 
    \end{itemize}
\end{frame}


\begin{frame}[fragile]
    \frametitle{Esempio di XML}
    \begin{verbatim}
<DataModelElement xsi:type="webml:Entity" id="0" >       
    <Attribute id="1" name="OID" type="OID" /> 
    <Attribute id="2" name="Name"/> 
    <Attribute id="3" name="Psw"/> 
    <Attribute id="4" name="Email" contentType=""/> 
</DataModelElement> 
    \end{verbatim}
\end{frame}

\subsection{XSLT}
\subsection{Esempio di output di XSLT}
\subsection{Document Type Definition}
\subsection{XML Schema}
\subsubsection{Esempio di XMLS}

\section{AJAX}
\subsection{Cos'è Ajax}

\begin{frame}
    \frametitle{Cos'è Ajax}
    \begin{itemize}
    \item \textbf{A}synchronous \textbf{J}avaScript \textbf{A}nd \textbf{X}ML
    \end{itemize}

\end{frame}

\subsection{Problemi e vantaggi derivanti da AJAX}
\subsection{XMLHttp} 

\end{document}